\documentclass[11pt,onecolumn]{article}
%\usepackage{naaclhlt2013}
\usepackage{acl2012-onecolumn}
\usepackage{times}
\usepackage{latexsym}
\usepackage{dsfont}
\usepackage{amsmath}
\usepackage{}
\usepackage{url}
\usepackage[pdftex]{graphicx}
\usepackage{color}
\usepackage[utf8]{inputenc}
\newcommand{\cjdcomment}[1]{\textcolor{red}{\bf\small [#1 --CJD]}}
\newcommand{\nascomment}[1]{\textcolor{green}{\bf\small [#1 --NAS]}}
\newcommand{\wacomment}[1]{\textcolor{blue}{\bf\small [#1 --WA]}}
\DeclareMathOperator*{\argmax}{arg\,max}
\setlength\titlebox{6.5cm}    % Expanding the titlebox

\newenvironment{itemizesquish}{\begin{list}{\labelitemi}{\setlength{\itemsep}{0em}\setlength{\labelwidth}{0.5em}\setlength{\leftmargin}{\labelwidth}\addtolength{\leftmargin}{\labelsep}}}{\end{list}}

\newcommand{\kinstring}[1]{\textsf{#1}}
\newcommand{\engstring}[1]{\texttt{\small{#1}}}


\title{Latent CRF}

\date{}

\begin{document}
\maketitle
\begin{abstract}
abstract...
\end{abstract}

\section{Problem}

Given a set of sentences \{\textbf{x}\}, find the most probable POS sequence \textbf{y} for each sentence. 

\section{Model}

$x_i$ is the $i^{th}$ word in \textbf{x}. $x^k$ refers to a unique type throughout the sentences. $x$ belongs to discrete domain $\cal{X}$. 
Similarly, $y_i$ is the $i^{th}$ POS tag in \textbf{y}. $y^k$ refers to a unique POS tag. $y$ belongs to discrete domain $\cal{Y}$. 
For each sentence, we introduce an additional sequence variable \textbf{z} which belongs to domain $\cal{X}^*$ (similar to $\mathbf{x}$).\\


\textbf{Generative story:}
\begin{itemizesquish}
\item Given a sentence \textbf{x}, generate a lossy compression \textbf{y} of the sentence.
\item Given a compression \textbf{y}, generate a sentence \textbf{z}.
\end{itemizesquish}

The best compression $\hat{\mathbf{y}}$ is the one that produces sentence \textbf{z} similar to the original \textbf{x} with highest probability, according to some model of $p(\mathbf{y} \mid \mathbf{x}, \mathbf{z})$:

\begin{eqnarray}
\hat{\mathbf{y}} = \argmax_\mathbf{y} p(\mathbf{y} \mid \mathbf{x}, \mathbf{z}) 
\end{eqnarray}

We choose a CRF model for the first part of the generative story $p(\mathbf{y} \mid \mathbf{x})$:

\begin{eqnarray}
p_\lambda(\mathbf{y}\mid\mathbf{x}) &=& \frac{f(\mathbf{y}, \mathbf{x})}{Z_\lambda(\mathbf{x})}; \\
f(\mathbf{y}, \mathbf{x}) &=& \exp \sum_{i=1}^{|\mathbf{y}|} \boldsymbol{\lambda}^\top \mathbf{h}(y_i, y_{i-1}, \mathbf{x}, i), \\
Z_\lambda(\mathbf{x}) &=& \sum_\mathbf{y} \exp \sum_{i=1}^{|\mathbf{y}|} \boldsymbol{\lambda}^\top \mathbf{h}(y_i, y_{i-1}, \mathbf{x}, i)
\end{eqnarray}

where $\mathbf{h}$ is a vector of feature functions, and $\boldsymbol{\lambda}$ is a vector of feature weights.\\

We choose the following parameterization for $p(\mathbf{z} \mid \mathbf{y})$:

\begin{eqnarray}
p_\theta(\mathbf{z} \mid \mathbf{y}) &=& \prod_{i=1}^{|\mathbf{z}|} p(z_i\mid y_i) \\
 &=& \prod_{i=1}^{|\mathbf{z}|} \theta_{z_i|y_i}
\end{eqnarray}

\section{Inference}
\label{sec:inference}
Given certain parameter values for $\boldsymbol{\lambda}$ and $\boldsymbol{\theta}$, we can compute the probability that an arbitrary compression \textbf{y} of sentence \textbf{x} regenerates $\mathbf{z} = \mathbf{x}$ as follows:

\begin{eqnarray}
p(\mathbf{y}\mid\mathbf{x}, \mathbf{z}) &=& \frac{p(\mathbf{y}, \mathbf{z} \mid \mathbf{x})}{\sum_{\mathbf{y'}} p(\mathbf{y'}, \mathbf{z} \mid \mathbf{x})}; \\
p(\mathbf{y}, \mathbf{z} \mid \mathbf{x}) &=& p_\lambda(\mathbf{y}\mid\mathbf{x}).p_\theta(\mathbf{z}\mid\mathbf{y}) \\
&=& \frac{1}{Z_\lambda(\mathbf{x})}f(\mathbf{y}, \mathbf{x}).p_\theta(\mathbf{z}\mid\mathbf{y})
\end{eqnarray}

Fortunately, those quantities are cheap to compute, assuming $|\cal{Y}|$ is small. $Z_\lambda(\mathbf{x})$ can be computed with dynamic programming~\cite{sha:03}. The term $A(\mathbf{x},\mathbf{z}) = \sum_{\mathbf{y'}} p(\mathbf{y'}, \mathbf{z} \mid \mathbf{x})$, $B(\mathbf{x}, \mathbf{z}, z^*, y^*)$, $C(\mathbf{x}, \mathbf{z})$, $D(\mathbf{x}, \mathbf{z}, k)$ and $F(\mathbf{x}, k)$ (defined hereafter) can also be effeciently computed in a similar fashion. We can use the Viterbi algorithm to find the most likely compression $\hat{\mathbf{y}}$. All inference algorithms here are quadratic in $|\cal{Y}|$ and linear in $|\mathbf{x}|$.

\section{Learning}

Our objective is to maximize the corpus ($\cal{T}$) log-likelihood of regenerating \textbf{z} from \textbf{x}, according to the generative story.

\begin{align}
\ell\ell(\boldsymbol{\lambda}, \boldsymbol{\theta}) &= \sum_{\langle\mathbf{x},\mathbf{z}=\mathbf{x}\rangle \in \cal{T}} \log p(\mathbf{z}\mid\mathbf{x}) \\
 &= \sum_{\langle\mathbf{x},\mathbf{z}=\mathbf{x}\rangle \in \cal{T}} \log \sum_{\mathbf{y}} p_\lambda(\mathbf{y}\mid\mathbf{x}).p_\theta(\mathbf{z}\mid\mathbf{y}) \\
 &= \sum_{\langle\mathbf{x},\mathbf{z}=\mathbf{x}\rangle \in \cal{T}} \log \sum_{\mathbf{y}} \frac{(\prod_{i=1}^{|\mathbf{y}|} \theta_{z_i\mid y_i}) \exp \boldsymbol{\lambda}^\top \sum_{i=1}^{|\mathbf{y}|} \mathbf{h}(y_i, y_{i-1}, \mathbf{x}, i)}{Z_\lambda(\mathbf{x})} \\
&= \sum_{\langle\mathbf{x},\mathbf{z}=\mathbf{x}\rangle \in \cal{T}} \log \sum_{\mathbf{y}} (\prod_{i=1}^{|\mathbf{y}|} \theta_{z_i\mid y_i}) \exp \boldsymbol{\lambda}^\top \sum_{i=1}^{|\mathbf{y}|} \mathbf{h}(y_i, y_{i-1}, \mathbf{x}, i) \nonumber \\
 &- \sum_{\langle\mathbf{x},\mathbf{z}=\mathbf{x}\rangle \in \cal{T}} \log \sum_{\mathbf{y}} \exp \boldsymbol{\lambda}^\top \sum_{i=1}^{|\mathbf{y}|} \mathbf{h}(y_i, y_{i-1}, \mathbf{x}, i) \\
 &= log C(\mathbf{x}, \mathbf{z}) - log Z_\lambda(\mathbf{x})
\end{align}

Unfortunately, there's no closed-form MLE solution for $\boldsymbol{\lambda}$ and $\boldsymbol{\theta}$ due to the latent variables. We will try to reduce the risk of getting stuck at a local optimum, but no guarantees. We explore a number of ways to optimize the parameters:

\subsection{Block coordinate gradient descent}

We iteratively alternate between optimizing subsets of parameters, holding the remaining parameters fixed. Convergence to a local optimum is guaranteed. We start with optimizing the multinomial parameters using the closed-form solution:

\begin{align}
\frac{\partial \ell\ell}{\partial \theta_{x^*|y^*}} &= \sum_{\langle\mathbf{x},\mathbf{z}=\mathbf{x}\rangle \in \cal{T} : \exists \mathit{j} : \mathit{z}^* = \mathit{z}_\mathit{j} } \frac{\sum_{\mathbf{y}:\exists q:y^*=y_q} (\prod_{i=1}^{|\mathbf{y}|} \theta_{z_i\mid y_i}) \exp \boldsymbol{\lambda}^\top \sum_{i=1}^{|\mathbf{y}|} \mathbf{h}(y_i, y_{i-1}, \mathbf{x}, i) }{\theta_{x^*|y^*} \sum_{\mathbf{y}} (\prod_{i=1}^{|\mathbf{y}|} \theta_{z_i\mid y_i}) \exp \boldsymbol{\lambda}^\top \sum_{i=1}^{|\mathbf{y}|} \mathbf{h}(y_i, y_{i-1}, \mathbf{x}, i)}
\end{align}

Using the method of Lagrange multipliers to enforce the constraint $\sum \theta_{.|y^l} = 1 \quad \forall l$, the Maximum Liklihood Estimate $\hat{\boldsymbol{\theta}}$ is:

\begin{align}
\hat{\theta}_{z^*|y^*} &= \frac{1}{m_{y^*}} \sum_{\langle\mathbf{x},\mathbf{z}=\mathbf{x}\rangle \in \cal{T}} \frac{\sum_{\mathbf{y}} (\prod_{i=1}^{|\mathbf{y}|} \theta_{z_i\mid y_i} e^{\boldsymbol{\lambda}^\top \mathbf{h}(y_i, y_{i-1}, \mathbf{x}, i)}) \sum_{i=1}^{|\mathbf{y}|}\delta_{y_i=y^*,z_i=z^*} }{\sum_{\mathbf{y}} (\prod_{i=1}^{|\mathbf{y}|} \theta_{z_i\mid y_i} e^{ \boldsymbol{\lambda}^\top \mathbf{h}(y_i, y_{i-1}, \mathbf{x}, i)}) } \\
&= \frac{1}{m_{y^*}} \sum_{\langle\mathbf{x},\mathbf{z}=\mathbf{x}\rangle \in \cal{T}} \frac{B(\mathbf{x}, \mathbf{z}, z^*, y^*)}{C(\mathbf{x}, \mathbf{z})}; \\
B(\mathbf{x}, \mathbf{z}, z^*, y^*) &= \sum_{\mathbf{y}} (\prod_{i=1}^{|\mathbf{y}|} \theta_{z_i\mid y_i} e^{\boldsymbol{\lambda}^\top \mathbf{h}(y_i, y_{i-1}, \mathbf{x}, i)}) \sum_{i=1}^{|\mathbf{y}|}\delta_{y_i=y^*,z_i=z^*}  \\
C(\mathbf{x}, \mathbf{z}) &= \sum_{\mathbf{y}} (\prod_{i=1}^{|\mathbf{y}|} \theta_{z_i\mid y_i} e^{ \boldsymbol{\lambda}^\top \mathbf{h}(y_i, y_{i-1}, \mathbf{x}, i)})
\end{align}

$m_{y^*}$ is the Lagrange multiplier and can be computed by normalizing $\sum \theta_{.|y^l}$ parameters to sum to one.\\

After updating the $\boldsymbol{\theta}$, we turn to $\boldsymbol{\lambda}$. We use L-BFGS to optimize these, which requires computing the gradient:

\begin{align}
  \frac{\partial \ell\ell}{\partial \lambda_k} &= \sum_{\langle\mathbf{x},\mathbf{z}=\mathbf{x}\rangle \in \cal{T}} 
  [
     \frac{\sum_{\mathbf{y}} (\prod_{i=1}^{|\mathbf{y}|} \theta_{z_i\mid y_i}) e^{\boldsymbol{\lambda}^\top \sum_{i=1}^{|\mathbf{y}|} \mathbf{h}(y_i, y_{i-1}, \mathbf{x}, i)} \sum_{j} h_k(y_j, y_{j-1}, \mathbf{x}, j) } 
          {\sum_{\mathbf{y}} (\prod_{i=1}^{|\mathbf{y}|} \theta_{z_i\mid y_i}) e^{\boldsymbol{\lambda}^\top \sum_{i=1}^{|\mathbf{y}|} \mathbf{h}(y_i, y_{i-1}, \mathbf{x}, i)} } \nonumber \\
     &\qquad \qquad \qquad - \frac{\sum_{\mathbf{y}} e^{\boldsymbol{\lambda}^\top \sum_{i=1}^{|\mathbf{y}|} \mathbf{h}(y_i, y_{i-1}, \mathbf{x}, i)} \sum_{j} h_k(y_j, y_{j-1}, \mathbf{x}, j)}
          {\sum_{\mathbf{y}} e^{\boldsymbol{\lambda}^\top \sum_{i=1}^{|\mathbf{y}|} \mathbf{h}(y_i, y_{i-1}, \mathbf{x}, i)} }
  ] \\
  &= \sum_{\langle\mathbf{x},\mathbf{z}=\mathbf{x}\rangle \in \cal{T}}
    \frac{D(\mathbf{x}, \mathbf{z}, k)}{C(\mathbf{x}, \mathbf{z})} - \frac{F(\mathbf{x}, k)}{Z_\lambda(\mathbf{x})}; \\
  D(\mathbf{x}, \mathbf{z}, k) &= \sum_{\mathbf{y}} (\prod_{i=1}^{|\mathbf{y}|} \theta_{z_i\mid y_i} e^{\boldsymbol{\lambda}^\top \mathbf{h}(y_i, y_{i-1}, \mathbf{x}, i)}) \sum_{j} h_k(y_j, y_{j-1}, \mathbf{x}, j) \\
  F(\mathbf{x}, k) &= \sum_{\mathbf{y}} \prod_{i=1}^{|\mathbf{y}|} e^{\boldsymbol{\lambda}^\top \mathbf{h}(y_i, y_{i-1}, \mathbf{x}, i)} \sum_{j} h_k(y_j, y_{j-1}, \mathbf{x}, j)
\end{align} 



\subsection{Expectation maximization}


\subsection{Stochastic gradient descent with softmax parameterization}


\subsection{Exponentiated gradient descent}




\end{document}
